\begin{titlepage}
	\BgThispage

	\newgeometry{left=1cm,right=2cm}
	\vspace*{2cm}
	\noindent
	\textcolor{white}{\midsf Comparative Analysis of \\ \vspace{2mm} \\ \bigsf Motif Finding Algorithms}
	\vspace*{0.9cm}\par
	\noindent
	\begin{minipage}{0.3\linewidth}
		\begin{flushright}
			\vspace*{40pt}
			\printauthor
		\end{flushright}
	\end{minipage} \hspace{15pt}
	%
	\begin{minipage}{0.02\linewidth}
		\textcolor{titlepagecolor}{\rule{3pt}{400pt}}
	\end{minipage} \hspace{-5pt}
	%
	\begin{minipage}{0.64\linewidth}
		\vspace{5pt}
		\begin{abstract}
			Motif finding, a fundamental problem in bioinformatics, involves identifying recurring patterns in biological sequences that often represent functional elements such as transcription factor binding sites. In this report, we present a comparative analysis of two prominent motif finding algorithms: Randomized Motif Search (RMS) and Gibbs Sampler. Additionally, we evaluate the effectiveness of two widely used motif discovery tools, MEME and STREME, which implement these algorithms.

			RMS employs a randomized strategy, iteratively searching for motifs within a set of sequences, while Gibbs Sampler adopts a probabilistic sampling approach to refine motif predictions based on the posterior distribution of motif occurrences. MEME leverages probabilistic modeling to identify motifs enriched in the input sequences, while STREME focuses on discovering motifs enriched in specific subsets of sequences.

			We experiment with different techniques and assess the performance of these methods over a number of different datasets based on computational efficiency and accuracy. Our comparative analysis provides insights into the strengths and limitations of each algorithm and tool in motif finding tasks.
		\end{abstract}
	\end{minipage}
\end{titlepage}
\restoregeometry
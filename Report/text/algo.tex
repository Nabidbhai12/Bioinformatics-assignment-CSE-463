\section{Algorithms}

We utilized two prominent motif finding algorithms, Randomized Motif Search (RMS) and Gibbs Sampler in our study. For both algorithms, we defined the score function as the sum of Hamming distances between the motifs and the consensus sequence.

\subsection{Randomized Motif Search (RMS)}

Randomized Motif Search (RMS) is a randomized algorithm that iteratively searches for motifs within a set of sequences. It is particularly useful for discovering short, recurring sequences that may represent regulatory motifs.

The steps of the RMS algorithm are as follows:

\begin{enumerate}
	\item \textbf{Initialization} \ Randomly select k-mers from each sequence in the dataset. This forms the initial set of motifs.
	\item \textbf{Motif Profile Construction} \ Construct a profile matrix from the selected motifs, consisting of the frequency of each nucleotide at each position in the motif.
	\item \textbf{Motif Prediction} \ Use the profile matrix to identify the most probable motif in each sequence.
	\item \textbf{Motif Set Update} \ Update the motif set by replacing one motif with a new one from the corresponding sequence.
	\item \textbf{Score Calculation} \ Calculate the score of the motif set using the profile matrix and the consensus sequence.
	\item Repeat steps 2 and 3 for a certain number of iterations or until the score converges.
	\item Return the motif set with the lowest score.
\end{enumerate}

Since the RMS algorithm initializes randomly, it may take a long time to converge to an optimal solution. In most cases, it fails to reach the global optimum, as the quality of the solution depends on the initial set of motifs. In practice, the algorithm is run multiple times with different initializations to improve the chances of finding the optimal solution.

The algorithm for RMS is as follows:

\begin{algorithm}[!h]
	\caption{Randomized Motif Search}\label{alg:rms}
	\KwData{A set of sequences $D$, motif length $k$, number of iterations $N$}
	\KwResult{A set of motifs $M$}
	$Motif \gets \text{Randomly select k-mers from each sequence in } D$\;
	$BestMotif \gets Motif$\;
	\For{$i \gets 1$ \KwTo $N$}{
		$P \gets Profile(Motif)$;
		$Motif \gets MotifPrediction(D, P)$\;
		\If{$Score(Motif) < Score(BestMotif)$}{
			$BestMotif \gets Motif$\;
		}
		\Else{
			\Return{$BestMotif$}
		}
	}
\end{algorithm}

\subsection{Gibbs Sampler}

Gibbs Sampler is a probabilistic sampling algorithm that refines motif predictions based on the posterior distribution of motif occurrences. It is particularly useful for identifying motifs in a set of sequences with variable lengths and background noise. It iteratively discards one l-mer at a time and replaces it with a new one, based on the probability distribution of the motif occurrences.

The steps of the Gibbs Sampler algorithm are as follows:

\begin{enumerate}
	\item \textbf{Initialization} \ Randomly select k-mers from each sequence in the dataset. This forms the initial set of motifs.
	\item \textbf{Sequence Selection} \ Randomly select one sequence from the dataset.
	\item \textbf{Motif Profile Construction} \ Construct a profile matrix from the  all but the selected motif, consisting of the frequency of each nucleotide at each position in the motif.
	\item \textbf{Motif Prediction} \ Use the profile matrix to identify the most probable motif in the selected sequence.
	\item \textbf{Motif Set Update} \ Update the motif set by replacing the selected motif with the most probable one from the corresponding sequence.
	\item \textbf{Score Calculation} \ Calculate the score of the motif set using the profile matrix and the consensus sequence.
	\item Repeat steps 2 to 5 for a certain number of iterations or until the score converges.
	\item Return the motif set with the lowest score.
\end{enumerate}

The algorithm for Gibbs Sampler is as follows:

\begin{algorithm}[!h]
	\caption{Gibbs Sampler}\label{alg:gibbs}
	\KwData{A set of sequences $D$, motif length $k$, number of iterations $N$}
	\KwResult{A set of motifs $M$}
	$Motif \gets \text{Randomly select k-mers from each sequence in } D$\;
	$BestMotif \gets Motif$\;
	\For{$j \gets 1$ \KwTo $N$}{
		$i \gets Random(|D|)$\;
		$P \gets Profile(Motif - Motif[i])$\;
		$Motif[i] \gets MotifPrediction(D[i], P)$\;
		\If{$Score(Motif) < Score(BestMotif)$}{
			$BestMotif \gets Motif$\;
		}
	}
	\Return{$BestMotif$}
\end{algorithm}
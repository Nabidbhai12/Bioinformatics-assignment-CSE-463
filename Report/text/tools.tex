\section{Software}

We use two widely used motif discovery tools, MEME and STREME. Both tools are open-source and availble in The MEME Suite \cite{memesuite}. We obtained the source code for the suite and compiled it on our local machine for uniform comparison between the tools.

\subsection{Installation}

We installed the MEME Suite on our local machine. We used the following commands to install it on Debian-based systems.

\begin{verbatim}
	$ wget https://meme-suite.org/meme/meme-software/5.5.5/meme-5.5.5.tar.gz
	$ tar -xvzf meme-5.5.5.tar.gz
	$ cd meme-5.5.5
	$ ./configure --enable-build-libxml2 --enable-build-libxslt
	$ make
	$ make test
	$ make install
\end{verbatim}


\subsection{MEME}

MEME, or Multiple EM for Motif Elicitation \cite{Bailey1994}, is a widely used motif discovery tool. It uses the Expectation Maximization (EM) algorithm to find motifs in a set of sequences. The tool also provides a statistical significance score for each motif.

\subsubsection*{Usage}

We used the following basic command to run MEME on our dataset.

\begin{verbatim}
	$ meme <input.fasta> -dna -w <width> -text
\end{verbatim}


\subsection{STREME}

STREME, or Sensitive, Thorough, Rapid, Enriched Motif Elicitation \cite{btab203}, is another motif discovery tool that uses the EM algorithm to find motifs in a set of sequences. It extends the capabilities of MEME by incorporating additional features tailored for regulatory motif analysis.

\subsubsection{Usage}

We used the following basic command to run STREME on our dataset.

\begin{verbatim}
	$ ./streme --p <input.fasta> -nmotifs 1 -minw <min-width>
			-maxw <max-width> -thresh <threshold> -dna anr -text
\end{verbatim}

\subsection{Scripts}

We wrote a couple of scripts to automate the process of running MEME and STREME on our dataset. The scripts are written in Bash and are as follows:

\subsubsection{memes\_run.sh}

\begin{lstlisting}[language=bash]
#!/bin/bash

# Directory where output files will be stored
output_dir1="hm03"
output_dir2="yst04r"
output_dir3="yst08r"

# Check if the output directory exists, if not, create it
if [ ! -d "$output_dir1" ]; then
  mkdir -p "$output_dir1"
fi
if [ ! -d "$output_dir2" ]; then
  mkdir -p "$output_dir2"
fi
if [ ! -d "$output_dir3" ]; then
  mkdir -p "$output_dir3"
fi

outfile_counter=10
for w in {8..16}; do
  ./meme hm03.fasta -dna -w $w -text >
    "${output_dir1}/out${outfile_counter}.txt"
  ((outfile_counter++))
done

outfile_counter=10
for w in {8..16}; do
  ./meme yst04r.fasta -dna -w $w -text >
    "${output_dir2}/out${outfile_counter}.txt"
  ((outfile_counter++))
done

outfile_counter=10
for w in {8..16}; do
  ./meme yst08r.fasta -dna -w $w -text >
    "${output_dir3}/out${outfile_counter}.txt"
  ((outfile_counter++))
done
\end{lstlisting}

\subsubsection{stremes\_run.sh}

\begin{lstlisting}[language=bash]
#!/bin/bash

# Directory where output files will be stored
output_dir1="hm03"
output_dir2="yst04r"
output_dir3="yst08r"

# Check if the output directory exists, if not, create it
if [ ! -d "$output_dir1" ]; then
  mkdir -p "$output_dir1"
fi
if [ ! -d "$output_dir2" ]; then
  mkdir -p "$output_dir2"
fi
if [ ! -d "$output_dir3" ]; then
  mkdir -p "$output_dir3"
fi

outfile_counter=1
for w in {8..16}; do
  ./streme --p hm03.fasta -nmotifs 1 -minw $w -maxw $w
    -thresh 0.05 -dna anr -text >
	"${output_dir1}/out${outfile_counter}.txt"
  ((outfile_counter++))
done

outfile_counter=1
for w in {8..16}; do
  ./streme --p yst04r.fasta -nmotifs 1 -minw $w -maxw $w
    -thresh 0.05 -dna anr -text >
    "${output_dir2}/out${outfile_counter}.txt"
  ((outfile_counter++))
done

outfile_counter=1
for w in {8..16}; do
  ./streme --p yst08r.fasta -nmotifs 1 -minw $w -maxw $w
    -thresh 0.05 -dna anr -text >
    "${output_dir3}/out${outfile_counter}.txt"
  ((outfile_counter++))
done


\end{lstlisting}